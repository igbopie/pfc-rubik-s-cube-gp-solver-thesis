%\bibliography{bib/biblio}
\chapter{Introducción}\label{ch:intro}

En este proyecto de fin de carrera se tiene como objetivo tratar el tema de la
programación genética. De esta forma se pretende generar un programa que resuelva
el cubo de Rubik de forma óptima. El motivo del proyecto se debe a una
competición ofrecida por Parabon, en la que se pretende participar. El ganador
de la competición tendrá la oportunidad de exponer su proyecto en \textit{The
Genetic and Evolutionary Computation Conference 2009} (GECCO 2009),
una famosa conferencia sobre la computación evolutiva.

La programación genética es una rama de la computación evolutiva donde se imita
el modelo de evolución natural aplicado a programas informáticos. La computación
evolutiva procede de difqerentes orígenes: programación evolutiva
\cite{Fogel:1966}, algoritmos genéticos \cite{Holland:1975}, estrategias de
evolución \cite{Rechenberg:1971,Schwefel:1975} y por último programación genética
\cite{Cramer:1985,Koza:1992}. Todas estas vertientes nacieron de forma
independiente, pero en los noventa se unificaron para formar la computación
evolutiva.  Todas estas ramas coinciden en inspirarse en la teoría de la
evolución moderna de la que Darwin \cite{Darwin:1859} plantó sus pilares, para
resolver problemas, normalmente de optimización, en la informática.

La computación evolutiva se caracteriza por encontrar soluciones inusuales a
problemas de los que no existe un método resolutivo claro o viable en términos
temporales. La potencia de estos sistemas reside en una multitudinaria
exploración aleatoria del espacio búsqueda.

Uno de los problemas del cubo de Rubik es que debido al gran número de
posibilidades de desorden no es posible determinar a priori un número mínimo de
pasos que se necesitan para resolver cualquier cubo. De este modo, los métodos
que han dado mejores resultados han sido los sistemas de búsquedas exhaustivas
informáticas,  batiendo el record con 26 pasos \cite{Cooperman-Kunkle:2007} y
22 pasos \cite{Rokicki:2008}. Sin embargo, explorar todo el extenso espacio de
cubos de Rubik posibles resulta un proceso muy costoso, por lo que es posible que
aún existan soluciones más cortas inexploradas actualmente. Esto hace que la
programación genética sea un desafío interesante para la resolución mediante la
computación evolutiva.

El resto de este documento trendrá la siguiente estructura. Un capítulo
introductorio donde se describirá de la teoría de la evolución en la que se
inspiran las principales ideas de los sistemas evolutivos
(capítulo \ref{ch:estado-arte}). Procederemos a adentrarnos en la  situación actual de los sistemas evolutivos, en concreto en la programación evolutiva, donde se verán todos los procesos existentes en una evolución informática. Además profundizaremos y
describiremos los métodos más utilizados en la programación evolutiva, con sus
ventajas y sus defectos.

Una vez introducidos en la materia de la programación genética fijaremos los
objetivos de este proyecto de fin de carrera (capítulo \ref{ch:objetivos}).

\begin{itemize}
  \item Desarrollo de una representación adecuada del problema para hacer posible su solución usando
  programación genética.
  \item Experimentación y ajuste de los parámetros del algoritmo evolutivo a fin de lograr soluciones con el
  mejor desempeño posible.
  \item Ejecución del sistema y extracción de la mejor solución para ser
  enviada a la competición.
\end{itemize}

En el capítulo \ref{ch:diseno-algoritmo} explicaremos el diseño de nuestro algoritmo que hemos
utilizado para la realización de este proyecto. Veremos las evoluciones que ha
sufrido nuestro lenguaje hasta llegar hasta el seleccionado para ser
implementado. Además hablaremos de todo el complejo proceso de evaluación y
asignación del fitness a nuestros individuos.

Una vez abordado el diseño del sistema, nos centraremos en la implementación
(capítulo \ref{ch:implementacion}). Explicaremos las herramientas que nos han
ayudado a desarrollar un programa evolutivo, además de las librerías que modelan
el cubo de Rubik computacionalmente. Terminaremos este capítulo explicando la
implementación con diagramas de clases y la configuración final de la
plataforma evolutiva.

En el capítulo \ref{ch:pruebasyresultados} hablaremos de las pruebas realizadas, explicando el
porqué de ellas y las conclusiones de su resultado.

Las conclusiones finales del proyecto se verán en el capitulo \ref{ch:concl}, además de
hablar de las futuras líneas de investigación.

Como anexos añadiremos la planificación del proyecto y un desarrollo en detalle
de los parámetros de utilización de ECJ.
